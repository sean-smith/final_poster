\title{Security Analysis of Github}
\author{
        Sean Smith, Kyle Holzinger, Amalia Safer\\
}
\date{\today}

\documentclass[12pt]{article}

\begin{document}
\maketitle

\section{Introduction}
This is time for all good men to come to the aid of their party!

\paragraph{Outline}
The remainder of this article is organized as follows.
Section~\ref{previous work} gives account of previous work.
Our new and exciting results are described in Section~\ref{results}.
Finally, Section~\ref{conclusions} gives the conclusions.

\section{Cookies}\label{cookies}
A session cookie called \tt{user\_session} is stored which contains a seemingly random nonce. When a get request is made to \tt{https://github.com}, the cookie is sent and the database is queried to see if the cookie is valid. If the cookie is valid it will return user data as if the user is logged in. There are two more cookies of importance \tt{logged\_in} which is a yes/no value and \tt{dotcom\_user} which is the user's username. To impersonate a user, only the \tt{user\_session} is needed, the \tt{logged\_in} cookie will always be yes and the \tt{dotcom\_user} will be filled by the server if the \tt{user\_session} cookie is valid. When the user logs out, the cookie is invalided by the server in a post request to \tt{https://github.com/logout} that contains the content type and an authenticity token. The authenticity token is a random nonce that github uses to prevent against CSRF. If an attacker is using the user's cookie and the user logs out, the attacker's cookie will be invalidated.

%Is there any way to get a new cookie without knowing the password? -- No only editing the cookie (which is the same as logging out)
%What the fuck is the authenticity token? I think it's either a digital signature of the post data, a random nonce or token that the server knows to prevent csrf.

When the user logs in, a post request is sent to \tt{https://github.com/session} with the username, password and authenticity token.

One attack is to guess a random cookie and query to see if it's valid. There are approximately 8 million active github users at a time so roughly 8 million valid cookies. Since you don't need the \tt{logged\_in} cookie to be set correctly, you can construct a random cookie and check if it's valid. The length of the cookie is 80 characters and each character is from the universe (a-z, A-Z, 0-9, -, \_) which has a size of 64. Say the set of correct cookies $S$  has size $|S| = 8,000,000$, the universe $U$ has a size of $|U| = 64^{80}$. The probability of getting a correct cookie is so low that it's not a reasonable attack.

$$Pr[c \in ] \approx$$

\section{Results}\label{results}
In this section we describe the results.

\section{Conclusions}\label{conclusions}
We worked hard, and achieved very little.

\bibliographystyle{abbrv}
\bibliography{main}

\end{document}
This is never printed